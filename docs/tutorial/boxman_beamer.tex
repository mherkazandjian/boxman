\documentclass[aspectratio=169]{beamer}

% --- Styling (simple + professional)
\usetheme{Boadilla}
\usecolortheme{default}
\setbeamertemplate{navigation symbols}{}
\setbeamertemplate{footline}[frame number]

\usepackage[T1]{fontenc}
\usepackage{lmodern}
\usepackage{microtype}
\usepackage{booktabs}
\usepackage{array}
\usepackage{listings}
\usepackage{xcolor}
\usepackage{hyperref}

% Listings setup
\lstdefinestyle{boxman}{
  basicstyle=\ttfamily\small,
  columns=fullflexible,
  breaklines=true,
  frame=single,
  rulecolor=\color{black!15},
  framerule=0.5pt,
  showstringspaces=false,
  upquote=true,
}

\title{Getting Started with Boxman}
\subtitle{Your first local VM cluster (libvirt/QEMU) in \textasciitilde 30 minutes}
\author{}
\date{}

\begin{document}

% --- Title
\begin{frame}[fragile]
  \titlepage
\end{frame}

% --- What is Boxman
\begin{frame}[fragile]{What is Boxman?}
\begin{itemize}
  \item Infrastructure-as-code for \textbf{local VM clusters} on \textbf{libvirt/QEMU}.
  \item Think: \textbf{``Docker Compose for VMs''}.
  \item One YAML file describes:\\
    \hspace*{1em}machines, networks, disks, and helper files.
  \item Boxman provisions everything: networks, clones, disks, SSH access.
\end{itemize}
\end{frame}

% --- Agenda
\begin{frame}[fragile]{What we will do (quick path)}
\begin{enumerate}
  \item Install libvirt + tools
  \item Create a base VM image (one-time)
  \item Install Boxman
  \item Write a minimal \texttt{conf.yml}
  \item Provision and SSH in
\end{enumerate}
\end{frame}

% --- Requirements
\begin{frame}[fragile]{Prerequisites}
\begin{columns}[T,onlytextwidth]
  \column{0.55\textwidth}
  \textbf{You need:}
  \begin{itemize}
    \item Linux host with libvirt support
    \item \textbf{8 GB RAM} minimum
    \item \textbf{~50 GB} free disk
    \item \textbf{Python 3.12} (3.13 not supported)
  \end{itemize}

  \column{0.45\textwidth}
  \textbf{System tools:}
  \begin{itemize}
    \item \texttt{virt-install}, \texttt{virt-clone}
    \item \texttt{virsh}, \texttt{qemu-img}
    \item \texttt{sshpass}
    \item user in \texttt{libvirt} (and \texttt{kvm}) group
  \end{itemize}
\end{columns}
\end{frame}

% --- Install libvirt
\begin{frame}[fragile]{Install libvirt (pick your distro)}
\textbf{Arch:}
\begin{lstlisting}[style=boxman]
sudo pacman -S libvirt qemu-full virt-install virt-clone sshpass
sudo systemctl enable --now libvirtd
sudo usermod -aG libvirt $USER
\end{lstlisting}

\textbf{Ubuntu/Debian:}
\begin{lstlisting}[style=boxman]
sudo apt update
sudo apt install -y libvirt-daemon-system libvirt-clients qemu-kvm virtinst sshpass bridge-utils python3-pip python3-venv
sudo systemctl enable --now libvirtd
sudo usermod -aG libvirt,kvm $USER
\end{lstlisting}
\vspace{0.5em}
\textit{Log out and back in after adding groups.}
\end{frame}

% --- Verify libvirt
\begin{frame}[fragile]{Verify libvirt works}
\begin{lstlisting}[style=boxman]
virsh -c qemu:///system list
\end{lstlisting}
\begin{itemize}
  \item If you see a table (even empty), you're good.
  \item If connection fails, check \texttt{systemctl status libvirtd}.
\end{itemize}
\end{frame}

% --- Base image concept
\begin{frame}[fragile]{Base VM image (one-time setup)}
\begin{itemize}
  \item Boxman \textbf{doesn't install an OS} from scratch.
  \item It \textbf{clones a pre-existing VM} (your \emph{base image}) for each new VM.
  \item Set it up once:
    \begin{itemize}
      \item install OS
      \item enable SSH
      \item set a password (used once for SSH key injection)
    \end{itemize}
\end{itemize}
\end{frame}

% --- Create base image example
\begin{frame}[fragile]{Create a base image (example: Rocky 9)}
\begin{lstlisting}[style=boxman]
sudo virt-install \
  --name rocky9 \
  --ram 2048 \
  --vcpus 2 \
  --disk path=/var/lib/libvirt/images/rocky9.qcow2,size=20 \
  --os-variant rocky9 \
  --cdrom ~/Downloads/Rocky-9-latest-x86_64-minimal.iso \
  --network network=default \
  --graphics vnc
\end{lstlisting}
\begin{itemize}
  \item Finish install via console, then shut down:
\end{itemize}
\begin{lstlisting}[style=boxman]
virsh -c qemu:///system shutdown rocky9
\end{lstlisting}
\end{frame}

% --- Install Boxman
\begin{frame}[fragile]{Install Boxman (in a virtualenv)}
\begin{lstlisting}[style=boxman]
python3.12 -m venv boxman_env
source boxman_env/bin/activate
pip install -r requirements.txt
python setup.py install
boxman --version
\end{lstlisting}
\begin{itemize}
  \item If \texttt{boxman} isn't found, your venv probably isn't active.
\end{itemize}
\end{frame}

% --- Config: mental model
\begin{frame}[fragile]{The config file: the mental model}
\begin{itemize}
  \item One YAML file describes everything.
  \item Three top-level keys are the backbone:
\end{itemize}
\begin{lstlisting}[style=boxman]
version: 1.0
project: myproject
provider:
  libvirt: { ... }
clusters:
  my_cluster: { ... }
\end{lstlisting}
\vspace{0.3em}
\begin{itemize}
  \item Bonus: the YAML is rendered as \textbf{Jinja2} before parsing.
\end{itemize}
\end{frame}

% --- Build first config: minimal
\begin{frame}[fragile]{Your first config (single VM + single NAT network)}
\small
\begin{lstlisting}[style=boxman]
---
version: 1.0
project: mylab
provider:
  libvirt:
    uri: qemu:///system
    use_sudo: True
    virt_install_cmd: '/usr/bin/python /usr/bin/virt-install'
    virt_clone_cmd:   '/usr/bin/python /usr/bin/virt-clone'
    virsh_cmd:        '/bin/virsh'
  verbose: True

clusters:
  cluster_1:
    workdir: ~/workspaces/mylab
    base_image: rocky9
    proxy_host: localhost
    admin_user: 'root'
    admin_pass: 'file://~/workspaces/mylab/admin_pass.txt'
    admin_key_name: id_ed25519_boxman
    ssh_config: ssh_config

    networks:
      lab_net:
        mode: nat
        mac: '52:54:00:00:00:01'
        ip:
          address: '192.168.100.1'
          netmask: '255.255.255.0'
          dhcp:
            range: { start: '192.168.100.2', end: '192.168.100.254' }
        enable: True
        autostart: True

    vms:
      lab01:
        hostname: lab01
        memory: 2048
        network_adapters:
          - name: adapter_1
            link_state: 'up'
            network_source: 'lab_net'
\end{lstlisting}
\end{frame}

% --- Admin pass
\begin{frame}[fragile]{Password handling (kept simple)}
\begin{itemize}
  \item The password is used \textbf{once} to inject an SSH key (then key-only).
  \item Three supported ways:
\end{itemize}
\vspace{0.5em}
\begin{tabular}{@{}p{0.22\textwidth}p{0.72\textwidth}@{}}
\toprule
\textbf{Method} & \textbf{Example} \\
\midrule
Plain text & \texttt{'mysecretpass'} \\
From file & \texttt{'file://\textasciitilde /secrets/pass.txt'} \\
Env var & \texttt{'\$\{env:BOXMAN\_ADMIN\_PASS\}'} \\
\bottomrule
\end{tabular}
\end{frame}

% --- Provision
\begin{frame}[fragile]{Provision the cluster}
\begin{lstlisting}[style=boxman]
boxman --conf ~/workspaces/mylab/conf.yml provision
\end{lstlisting}
\textbf{Behind the scenes, Boxman:}
\begin{itemize}
  \item renders Jinja2 \textrightarrow{} YAML
  \item creates libvirt networks + rules
  \item clones VMs from the base image
  \item generates SSH keys + \texttt{ssh\_config}
  \item waits for DHCP IPs, then injects the public key
\end{itemize}
\end{frame}

% --- Verify + connect
\begin{frame}[fragile]{Verify and SSH in}
\textbf{Check the VM is running:}
\begin{lstlisting}[style=boxman]
virsh -c qemu:///system list
\end{lstlisting}
\textbf{SSH using the generated config:}
\begin{lstlisting}[style=boxman]
ssh -F ~/workspaces/mylab/ssh_config lab01
\end{lstlisting}
\begin{itemize}
  \item The SSH config maps \texttt{lab01} \textrightarrow{} IP and key.
\end{itemize}
\end{frame}

% --- Day-to-day
\begin{frame}[fragile]{Day-to-day commands (the ones you actually use)}
\begin{columns}[T,onlytextwidth]
  \column{0.52\textwidth}
  \textbf{Lifecycle}
  \begin{lstlisting}[style=boxman]
boxman --conf conf.yml provision
boxman --conf conf.yml deprovision
boxman --conf conf.yml list
  \end{lstlisting}

  \column{0.48\textwidth}
  \textbf{Power + snapshots}
  \begin{lstlisting}[style=boxman]
boxman --conf conf.yml start
boxman --conf conf.yml suspend pause
boxman --conf conf.yml suspend resume
boxman --conf conf.yml snapshot create -n snap1
boxman --conf conf.yml snapshot restore -n snap1
  \end{lstlisting}
\end{columns}
\end{frame}

% --- Multi-VM idea
\begin{frame}[fragile]{Going further: multi-VM + multi-network}
\begin{itemize}
  \item A common pattern: \textbf{public NAT} + \textbf{private backend}.
  \item Example idea:
    \begin{itemize}
      \item \texttt{web01} has 2 NICs (public + backend)
      \item \texttt{db01} is backend-only (isolated)
      \item add a dedicated data disk for the DB (e.g., \texttt{/dev/vdb})
    \end{itemize}
  \item Keep it boring: name networks clearly, avoid subnet conflicts.
\end{itemize}
\end{frame}

% --- Advanced features
\begin{frame}[fragile]{Advanced features (when you want less copy-paste)}
\textbf{Jinja2 templating inside YAML:}
\begin{itemize}
  \item Generate many similar VMs with loops.
  \item Pull secrets from environment variables.
  \item Reuse patterns for disks and NICs.
\end{itemize}
\vspace{0.5em}
\begin{lstlisting}[style=boxman]
vms:
  
  node{{ "%02d" % i }}:
    hostname: node{{ "%02d" % i }}
    memory: 2048
    network_adapters:
      - name: adapter_1
        link_state: 'up'
        network_source: 'my_network'
  
\end{lstlisting}
\end{frame}

% --- Troubleshooting
\begin{frame}[fragile]{Troubleshooting (fast checks)}
\begin{itemize}
  \item \textbf{Permission denied / can't connect to libvirt}
    \begin{itemize}
      \item Ensure you're in \texttt{libvirt} (and \texttt{kvm}) groups
      \item Try: \texttt{virsh -c qemu:///system list}
    \end{itemize}

  \item \textbf{Hangs at ``Waiting for IP address''}
    \begin{itemize}
      \item DHCP/client missing in base image
      \item Check DHCP range + base VM boots
    \end{itemize}

  \item \textbf{SSH key injection fails (missing sshpass)}
    \begin{itemize}
      \item Install \texttt{sshpass} for your distro
    \end{itemize}
\end{itemize}
\end{frame}

% --- Wrap
\begin{frame}[fragile]{Wrap-up}
\begin{itemize}
  \item Boxman shines when you want repeatable local labs.
  \item Start with one VM, then grow into multi-network setups.
  \item Use Jinja2 templating when configs start to feel copy-pasted.
\end{itemize}
\vspace{0.8em}
\textbf{One command to reset everything:}
\begin{lstlisting}[style=boxman]
boxman --conf conf.yml deprovision
\end{lstlisting}
\end{frame}

\end{document}
